%  ========================================================================
%  Copyright (c) 1985 The University of Washington
%
%  Licensed under the Apache License, Version 2.0 (the "License");
%  you may not use this file except in compliance with the License.
%  You may obtain a copy of the License at
%
%      http://www.apache.org/licenses/LICENSE-2.0
%
%  Unless required by applicable law or agreed to in writing, software
%  distributed under the License is distributed on an "AS IS" BASIS,
%  WITHOUT WARRANTIES OR CONDITIONS OF ANY KIND, either express or implied.
%  See the License for the specific language governing permissions and
%  limitations under the License.
%  ========================================================================
%

% Documentation for University of Washington thesis LaTeX document class
% by Jim Fox
% fox@washington.edu
%
%    Revised 2020/02/24, added \caption()[]{} option.  No ToC.
%
%    Revised for version 2015/03/03 of uwthesis.cls
%    Revised, 2016/11/22, for cleanup of sample copyright and title pages
%
%    This document is contained in a single file ONLY because
%    I wanted to be able to distribute it easily.  A real thesis ought
%    to be contained on many files (e.g., one for each chapter, at least).
%
%    To help you identify the files and sections in this large file
%    I use the string '==========' to identify new files.
%
%    To help you ignore the unusual things I do with this sample document
%    I try to use the notation
%       
%    % --- sample stuff only -----
%    special stuff for my document, but you don't need it in your thesis
%    % --- end-of-sample-stuff ---


%    Printed in twoside style now that that's allowed
%
 
\documentclass [11pt, proquest] {uwthesis}[2020/02/24]
 
%
% The following line would print the thesis in a postscript font 

\usepackage{booktabs}     % Professional tables
\usepackage{graphicx}     % For figures
\usepackage{amsmath}      % Math equations
\usepackage{hyperref}     % Hyperlinks in PDF
\usepackage[colorinlistoftodos,prependcaption]{todonotes}  % Drafting notes
% Usage: \todo{Note text} for margin notes
%        \todo[inline]{Note text} for inline notes
%        \listoftodos to generate list of all todos

\setcounter{tocdepth}{1}  % Print the chapter and sections to the toc
 

% ==========   Local defs and mods
%

% --- sample stuff only -----
% These format the sample code in this document

\usepackage{alltt}  % 
\newenvironment{demo}
  {\begin{alltt}\leftskip3em
     \def\\{\ttfamily\char`\\}%
     \def\{{\ttfamily\char`\{}%
     \def\}{\ttfamily\char`\}}}
  {\end{alltt}}
 
% metafont font.  If logo not available, use the second form
%
% \font\mffont=logosl10 scaled\magstep1
\let\mffont=\sf
% --- end-of-sample-stuff ---
 



\begin{document}
 
% ==========   Preliminary pages
%
% ( revised 2012 for electronic submission )
%

\prelimpages
 
%
% ----- copyright and title pages
%
\Title{Cross-Detector Descriptor Fusion: \\
Scale Control and Spatial Alignment for Local Feature Matching}
\Author{Frank Sossi}
\Year{2026}
\Program{Computer Science \& Software Engineering}

\Chair{Professor Clark Olson}{Committee Chair}{Computing \& Software Systems}
\Signature{Professor Min Chen}
\Signature{Professor Dong Si}

\copyrightpage

\titlepage  

 
%
% ----- signature and quoteslip are gone
%

%
% ----- abstract
%


\setcounter{page}{-1}
\abstract{%
% TODO: Check abstract length against UW requirements (currently ~340 words, typical limit 150-350)
% TODO: Verify all numbers match Results chapter
Local feature descriptors are fundamental to many computer vision applications including SLAM, structure-from-motion, and image retrieval. This thesis investigates two complementary approaches to improving local feature matching: using multiple detectors as a quality filter for keypoint selection, and fusing complementary descriptors to combine their strengths.

We present four main contributions. First, we demonstrate that spatial intersection between distinct keypoint detectors acts as a powerful quality filter. When different detection methods---whether SIFT and SURF or SIFT and KeyNet---both identify a keypoint at the same location, this consensus indicates a genuinely distinctive feature. Descriptors computed at intersection keypoints consistently outperform those on single-detector sets, with HardNet achieving 82.4\% mAP on SIFT-KeyNet intersection---a 25\% relative improvement and the best single-descriptor result in our study.

Second, we create a color version of the HPatches patch benchmark, enabling evaluation of color-aware descriptors. Using this dataset, we show that fusing the color histogram descriptor HoNC with learned CNN descriptors yields substantial improvements: HoNC+SOSNet concatenation achieves 50.6\% mAP on patch matching, outperforming all individual descriptors. HoNC's strong discriminative capability (high verification-to-matching ratio) complements the CNN's matching-optimized representations.

Third, we show that cross-family fusion (SIFT+CNN) requires pre-fusion L2 normalization to ensure equal contribution from each descriptor; with proper normalization, SIFT+HardNet achieves 46.0\% mAP on patches. Fourth, we show that keypoint scale is a dominant factor: filtering to large-scale keypoints yields 39\% relative improvement for SIFT and 21\% for CNN descriptors.

We develop DescriptorWorkbench, an open-source evaluation framework, and conduct over 100 experiments. The results demonstrate that keypoint quality---determined by detector consensus and scale---has greater impact on matching performance than descriptor algorithm choice alone.
}
 
%
% ----- contents & etc.
%
\tableofcontents
\listoffigures
\listoftables

% TODO: Remove \listoftodos before final submission
\listoftodos
 
%
% ----- acknowledgments
%
\acknowledgments{% \vskip2pc
  % {\narrower\noindent
  I would like to thank Professor Clark Olson for his guidance and support throughout this research.
  % \par}
}

%
% ----- dedication
%
% \dedication{\begin{center}to my dear wife, Joanna\end{center}}

%
% end of the preliminary pages
 
 
 
%
% ==========      Text pages
%

\textpages
 
% ========== Chapter 1
\chapter{Introduction}
\label{chap:introduction}
% Chapter 1: Introduction
% CLAUDE EDIT: Complete rewrite 2025-12-07

Local feature detection and description remain fundamental to many computer vision applications, including simultaneous localization and mapping (SLAM), structure-from-motion (SfM), image retrieval, and visual place recognition. Despite significant advances in end-to-end learned approaches, local feature methods remain competitive due to their interpretability, efficiency, and ability to handle wide-baseline matching.

\section{Motivation}

The computer vision community has developed two distinct families of local feature descriptors:

\begin{itemize}
    \item \textbf{Traditional descriptors} such as SIFT~\cite{lowe2004distinctive}, SURF~\cite{bay2006surf}, and their variants rely on hand-crafted gradient histograms computed at keypoint locations detected by scale-space analysis.

    \item \textbf{Learned descriptors} such as HardNet~\cite{mishchuk2017working} and SOSNet~\cite{tian2019sosnet} use convolutional neural networks trained on patch correspondence tasks, typically paired with learned detectors like KeyNet~\cite{barroso2019key}.
\end{itemize}

Each family has distinct characteristics. Traditional descriptors are well-understood, deterministic, and compute patches around keypoints detected at multiple scales. Learned descriptors achieve higher matching accuracy but are trained end-to-end with specific detector choices. A natural question arises: can we combine the complementary strengths of both families through descriptor fusion?

\section{Research Questions}

This thesis investigates the following research questions:

\begin{enumerate}
    \item \textbf{RQ1: Scale Impact.} How does keypoint scale distribution affect descriptor matching performance? Is larger scale universally better?

    \item \textbf{RQ2: Detector Agreement.} Does spatial agreement between different keypoint detectors (SIFT and KeyNet) provide a quality signal for filtering unreliable keypoints?

    \item \textbf{RQ3: Cross-Family Fusion.} Can traditional (SIFT) and learned (HardNet) descriptors be successfully fused? If not, why?

    \item \textbf{RQ4: Fusion Strategy.} What fusion strategy (averaging vs. concatenation) works best, and under what conditions?
\end{enumerate}

\section{Contributions}

This thesis makes the following contributions:

\begin{enumerate}
    \item \textbf{Scale Control Methodology.} We demonstrate that filtering keypoints by scale dramatically improves matching performance: +39\% relative for SIFT-family descriptors and +21\% relative for CNN descriptors. This finding suggests that keypoint quality is as important as descriptor algorithm choice.

    \item \textbf{Spatial Intersection Algorithm.} We develop a mutual nearest neighbor algorithm for establishing 1-to-1 correspondence between keypoints from different detectors, enabling cross-detector descriptor fusion with configurable spatial tolerance.

    \item \textbf{Distribution Incompatibility Analysis.} We identify and explain why SIFT+CNN descriptor fusion fails: the incompatible value distributions (non-negative versus zero-centered) cause averaging to destroy learned representations.

    \item \textbf{Successful CNN Fusion.} We show that CNN+CNN fusion succeeds when distributions are compatible, achieving 93.4\% mAP with HardNet+SOSNet concatenation---the highest result in our study.

    \item \textbf{DescriptorWorkbench Framework.} We develop an open-source evaluation framework implementing the metrics from Bojanic et al.~\cite{bojanic2020comparison}.
\end{enumerate}

\section{Thesis Organization}

The remainder of this thesis is organized as follows:

\begin{itemize}
    \item \textbf{Chapter~\ref{chap:background}} reviews local feature detection and description, covering both traditional and learned approaches, as well as prior work on descriptor fusion.

    \item \textbf{Chapter~\ref{chap:methodology}} describes our spatial intersection methodology, scale-matching strategy, and evaluation framework.

    \item \textbf{Chapter~\ref{chap:results}} presents experimental results from 108 experiments on the HPatches benchmark, analyzing scale control, fusion strategies, and viewpoint/illumination breakdown.

    \item \textbf{Chapter~\ref{chap:discussion}} interprets the results, explaining why certain fusion strategies succeed or fail based on distribution analysis.

    \item \textbf{Chapter~\ref{chap:conclusion}} summarizes findings and discusses future work directions.
\end{itemize}


% ========== Chapter 2
\chapter{Background and Related Work}
\label{chap:background}
% Chapter 2: Background and Related Work
% DRAFTING: New background chapter 2025-12-07

This chapter reviews the foundations of local feature detection and description, covering traditional and learned approaches, descriptor fusion methods, and evaluation benchmarks.

\section{Local Feature Detection}

\subsection{Traditional Detectors}

The SIFT (Scale-Invariant Feature Transform) detector~\cite{lowe2004distinctive} identifies keypoints by searching for scale-space extrema in a Difference-of-Gaussian (DoG) pyramid. Key characteristics include:

\begin{itemize}
    \item \textbf{Scale invariance}: Keypoints are detected across multiple octaves, with scale encoded in the keypoint metadata
    \item \textbf{Orientation assignment}: Dominant gradient orientation is computed for rotation invariance
    \item \textbf{Sub-pixel localization}: Taylor expansion refines keypoint position
\end{itemize}

Other traditional detectors include Harris corners, FAST, and SURF~\cite{bay2006surf}, each with different trade-offs between repeatability and computational cost.

\subsection{Learned Detectors}

KeyNet~\cite{barroso2019key} represents a hybrid approach combining hand-crafted and learned filters:

\begin{itemize}
    \item \textbf{Handcrafted filters}: Provide anchor structure for stable detection
    \item \textbf{Learned filters}: Trained to localize, score, and rank repeatable features
    \item \textbf{Multi-scale}: Operates on image pyramids similar to SIFT
\end{itemize}

KeyNet was designed specifically for pairing with learned descriptors like HardNet and SOSNet, achieving state-of-the-art repeatability on HPatches.

\subsection{End-to-End Learned Pipelines}

While KeyNet focuses on detection to be paired with separate descriptors, other approaches integrate detection and description into a single end-to-end trainable network:

\begin{itemize}
    \item \textbf{SuperPoint}~\cite{detone2018superpoint}: A fully convolutional network trained using self-supervision. It employs a single shared encoder and two separate decoder heads for interest point detection and descriptor generation.
    \item \textbf{LF-Net}~\cite{ono2018lf}: Learns local features without human supervision by optimizing a keypoint correspondence objective, enforcing geometry constraints across multi-view images.
    \item \textbf{ALIKE}~\cite{zhao2023aliked}: A lightweight framework that combines the strengths of handcrafted and learned methods, using a deformable transformation module to improve geometric invariance while maintaining high computational efficiency.
\end{itemize}

These end-to-end methods offer a compelling alternative to the detect-then-describe pipeline, though they often require strictly coupled detector-descriptor pairs, unlike the mix-and-match flexibility we investigate with KeyNet and SIFT.

\subsection{Detector Characteristics}

A critical observation for this thesis is that different detectors produce keypoints with different scale distributions:

\begin{itemize}
    \item \textbf{SIFT detector}: Produces predominantly small-scale keypoints (average 4.45 pixels in our experiments)
    \item \textbf{KeyNet detector}: Produces larger-scale keypoints (average 49.83 pixels), approximately 10$\times$ larger
\end{itemize}

This scale difference has significant implications for descriptor quality, as larger patches contain more distinctive information.

\section{Local Feature Descriptors}

\subsection{Traditional Descriptors}

\subsubsection{SIFT Descriptor}

The SIFT descriptor~\cite{lowe2004distinctive} computes a 128-dimensional vector from a 16$\times$16 pixel patch:

\begin{enumerate}
    \item Divide patch into 4$\times$4 grid of cells
    \item Compute 8-bin gradient orientation histogram per cell
    \item Concatenate to form 4$\times$4$\times$8 = 128 dimensions
    \item L2 normalize, clip values $>$ 0.2, re-normalize
\end{enumerate}

The resulting descriptor is \textit{non-negative} with values typically in [0, 0.3].

\subsubsection{Domain-Size Pooling (DSP-SIFT)}

Dong and Soatto~\cite{dong2015domain} introduced Domain-Size Pooling (DSP), which aggregates SIFT descriptors computed at multiple scales around each keypoint:

\begin{equation}
    d_{\text{DSP}} = \frac{1}{N} \sum_{i=1}^{N} d_{\sigma_i}
\end{equation}

where $d_{\sigma_i}$ is the SIFT descriptor computed at scale $\sigma_i$. DSP improves matching accuracy by capturing multi-scale information.

\subsubsection{RootSIFT}

RootSIFT applies element-wise square root after L1 normalization, which is equivalent to using the Hellinger kernel:

\begin{equation}
    d_{\text{RootSIFT}} = \sqrt{d_{\text{SIFT}} / \|d_{\text{SIFT}}\|_1}
\end{equation}

This transformation improves matching performance, particularly for illumination changes.

\subsection{Learned Descriptors}

\subsubsection{HardNet}

HardNet~\cite{mishchuk2017working} is trained using hard negative mining with triplet loss:

\begin{equation}
    L = \max(0, m + d(a, p) - d(a, n^-))
\end{equation}

where $(a, p)$ is a matching pair, $n^-$ is the hardest negative in the batch, and $m$ is the margin. The resulting 128-dimensional descriptor is \textit{zero-centered} with values typically in [-0.3, +0.3].

\subsubsection{SOSNet}

SOSNet (Second Order Similarity Network)~\cite{tian2019sosnet} extends HardNet by incorporating second-order information:

\begin{equation}
    L_{\text{SOS}} = L_{\text{triplet}} + \lambda L_{\text{second-order}}
\end{equation}

SOSNet achieves similar performance to HardNet with slightly better generalization.

\subsection{Descriptor Magnitude and Normalization}
% DRAFTING: Updated to reflect magnitude mismatch finding, not distribution incompatibility

A critical consideration for descriptor fusion is \textit{magnitude matching}. Different descriptor families produce values at vastly different scales before normalization:

\begin{table}[h]
\centering
\caption{Descriptor magnitude characteristics (before and after L2 normalization)}
\label{tab:magnitudes}
\begin{tabular}{lrrl}
\toprule
\textbf{Descriptor} & \textbf{Raw Range} & \textbf{After L2 Norm} & \textbf{Notes} \\
\midrule
SIFT & [0, 512] & [0, 0.3] & Gradient histogram counts \\
RootSIFT & [0, 22] & [0, 0.4] & After sqrt transform \\
HoNC & [0, 1] & [0, 0.3] & Normalized color histogram \\
HardNet & [-0.3, +0.3] & [-0.3, +0.3] & Trained with L2 output \\
SOSNet & [-0.3, +0.3] & [-0.3, +0.3] & Trained with L2 output \\
\bottomrule
\end{tabular}
\end{table}

When fusing descriptors from different families, \textbf{magnitude mismatch} can cause one descriptor to dominate distance calculations. For example, raw SIFT values (0--512) would overwhelm HardNet values (--0.3 to +0.3) in a concatenated descriptor. The solution is to L2 normalize each component \textit{before} fusion, ensuring equal contribution regardless of original magnitude.

\section{Descriptor Fusion Approaches}

\subsection{Early Fusion (Feature-Level)}

Early fusion combines descriptors before matching:

\textbf{Concatenation:}
\begin{equation}
    d_{\text{concat}} = [d_A, d_B]
\end{equation}

\textbf{Weighted Averaging:}
\begin{equation}
    d_{\text{avg}} = \alpha \cdot d_A + (1 - \alpha) \cdot d_B
\end{equation}

Both approaches require spatial alignment of keypoints when descriptors come from different detectors.

\subsection{Late Fusion (Score-Level)}

Late fusion combines matching scores rather than descriptors:

\begin{equation}
    s_{\text{fused}} = \alpha \cdot s_A + (1 - \alpha) \cdot s_B
\end{equation}

This approach does not require keypoint alignment but cannot create a unified descriptor representation.

\subsection{Research Gap}

Prior work has explored:
\begin{itemize}
    \item Detector-descriptor pairing studies showing mismatch penalties
    \item Late fusion of matching scores (Dempster-Shafer theory)
    \item Ensemble methods in image matching challenges
\end{itemize}

However, \textit{cross-detector early fusion}---averaging or concatenating descriptors from keypoints detected by different methods---remains unexplored. This thesis addresses this gap through our spatial intersection methodology.

\section{Evaluation Benchmarks}

\subsection{HPatches Dataset}

The HPatches benchmark~\cite{balntas2017hpatches} provides:
\begin{itemize}
    \item 116 sequences with ground-truth homographies
    \item 59 viewpoint sequences (geometric changes)
    \item 57 illumination sequences (photometric changes)
    \item Pre-extracted 65$\times$65 grayscale patches (original benchmark)
    \item Full images for keypoint-based evaluation
\end{itemize}

\subsection{Two Evaluation Protocols}
% DRAFTING: Added to distinguish patch vs full-image task protocols

We employ two distinct evaluation protocols, each with three tasks. The protocols differ in whether keypoint detection is part of the evaluation.

\subsubsection{Original HPatches Patch Protocol (Balntas et al.)}

The original HPatches benchmark~\cite{balntas2017hpatches} evaluates descriptors on \textit{pre-extracted patches}, removing keypoint detection as a variable:

\textbf{Patch Matching}: For each reference patch, rank all target patches from the same sequence by descriptor distance. Report mAP based on whether the correct correspondence ranks first.

\textbf{Patch Verification}: Binary classification of patch pairs as ``same location'' (positive) or ``different location'' (negative). Negatives include both same-sequence and different-sequence patches. Report AP.

\textbf{Patch Retrieval}: Given a query patch, rank a gallery containing true matches and distractors from different sequences. Report mAP.

We use this protocol for our \textbf{color patch benchmark} (Section~\ref{sec:patch_results}) to isolate descriptor fusion effects.

\subsubsection{Bojanic et al. Full-Image Protocol}

Bojanic et al.~\cite{bojanic2020comparison} define an evaluation protocol for \textit{full images with detected keypoints}:

\textbf{Image Matching}: Match descriptors between image pairs using the Second Nearest Neighbor (SNN) ratio test. A match is correct if the geometric reprojection error is below threshold. Report mAP.

\textbf{Keypoint Verification}: Binary classification distinguishing true correspondences from distractors sampled from \textit{other sequences} (not same-sequence negatives).

\textbf{Keypoint Retrieval}: Three-tier ranking with labels $y \in \{-1, 0, +1\}$: true positives (+1), hard negatives from the same sequence (0, ignored in scoring), and distractors from other sequences (-1).

We use this protocol for our \textbf{full-image experiments} (Sections~\ref{sec:baseline_results}--\ref{sec:intersection_results}) to study detector effects.

\subsubsection{Key Differences}

\begin{table}[h]
\centering
\caption{Comparison of evaluation protocols}
\label{tab:protocols}
\begin{tabular}{lll}
\toprule
\textbf{Aspect} & \textbf{Patch Protocol} & \textbf{Bojanic Protocol} \\
\midrule
Input & Pre-extracted patches & Full images \\
Keypoint detection & Not evaluated & Part of pipeline \\
Verification negatives & Same + different sequence & Different sequence only \\
Retrieval labeling & Binary (pos/distractor) & Three-tier ($-1, 0, +1$) \\
Isolates & Descriptor quality & Detector + descriptor \\
\bottomrule
\end{tabular}
\end{table}

We implement both protocols in DescriptorWorkbench to enable controlled experiments.


% ========== Chapter 3
\chapter{Methodology}
\label{chap:methodology}
% Chapter 3: Methodology
% CLAUDE EDIT: Comprehensive methodology chapter 2025-12-07

This chapter describes our methodology for cross-detector descriptor fusion, including the spatial intersection algorithm, scale-matching strategy, and evaluation framework.

\section{DescriptorWorkbench Framework}

We developed DescriptorWorkbench, an open-source evaluation framework for local feature descriptor research. The framework provides:

\begin{itemize}
    \item \textbf{Modular architecture}: Pluggable descriptor extractors, pooling strategies, and matchers
    \item \textbf{Database storage}: SQLite-based experiment tracking with comprehensive metrics
    \item \textbf{YAML configuration}: Declarative experiment specification
    \item \textbf{CLI tools}: \texttt{experiment\_runner} for evaluation, \texttt{keypoint\_manager} for keypoint set operations
\end{itemize}

\subsection{Supported Descriptors}

The framework implements the following descriptor types, utilizing a hybrid architecture of OpenCV for traditional methods and LibTorch for deep learning models:

\begin{table}[h]
\centering
\caption{Descriptor implementations in DescriptorWorkbench}
\label{tab:descriptors}
\begin{tabular}{llrl}
\toprule
\textbf{Type} & \textbf{Family} & \textbf{Dim} & \textbf{Backend} \\
\midrule
SIFT & Traditional & 128 & OpenCV \\
RootSIFT & Traditional & 128 & OpenCV + transform \\
DSP-SIFT v2 & Traditional & 128 & Custom (pyramid-aware) \\
RGBSIFT & Traditional & 384 & Custom (per-channel) \\
SURF & Traditional & 64/128 & OpenCV \\
HardNet & Learned & 128 & LibTorch (C++) \\
SOSNet & Learned & 128 & LibTorch (C++) \\
L2-Net & Learned & 128 & LibTorch (C++) \\
Composite & Fusion & varies & Custom \\
\bottomrule
\end{tabular}
\end{table}

\subsection{Database Schema}

Experiment results are stored in SQLite with the following key tables:

\begin{itemize}
    \item \texttt{experiments}: Descriptor type, keypoint set, parameters
    \item \texttt{results}: mAP, HP-V/HP-I breakdown, verification/retrieval metrics
    \item \texttt{keypoint\_sets}: Named keypoint collections with metadata
    \item \texttt{keypoints}: Individual keypoint records with coordinates and scale
\end{itemize}

\subsection{Deep Learning Integration}

A significant engineering challenge in this work was integrating research-grade deep learning models (typically implemented in Python/PyTorch) into a high-performance C++ evaluation pipeline to ensure fair comparison with optimized traditional descriptors like SIFT.

\subsubsection{Hybrid Architecture}

We developed a hybrid pipeline that leverages the best tools for each stage:

\begin{itemize}
    \item \textbf{Keypoint Detection (Python/Kornia)}: KeyNet detection is performed offline using Python scripts that interface with the Kornia library. This ensures 100\% fidelity to the reference implementation and avoids reimplementation errors. Coordinates are serialized to the SQLite database for reuse.
    \item \textbf{Descriptor Extraction (C++/LibTorch)}: For descriptor extraction, which must occur in the inner loop of matching experiments, we integrated the LibTorch (PyTorch C++) frontend directly into our application.
\end{itemize}

\subsubsection{Rejection of ONNX}

Initially, we attempted to deploy models using the ONNX (Open Neural Network Exchange) format run via OpenCV's DNN module. However, this approach proved unsuitable for our needs:
\begin{itemize}
    \item \textbf{Operator Incompatibility}: Key models like HardNet use specific normalization layers (e.g., \texttt{InstanceNorm}) and control structures that were not fully supported by the OpenCV ONNX importer at the time of development.
    \item \textbf{Performance Stability}: We observed inconsistent behavior and occasional crashes with complex graphs exported from newer PyTorch versions.
\end{itemize}

Consequently, we deprecated the ONNX pipeline in favor of TorchScript. Models are exported to the \texttt{.pt} format and loaded via \texttt{torch::jit::load} within our C++ \texttt{LibTorchWrapper}. This provides native PyTorch execution speed and guarantees numerical equivalence to the Python training code, while operating within the memory space of our C++ benchmarking tool.

\section{Spatial Intersection Algorithm}
\label{sec:intersection_algorithm}

To enable cross-detector descriptor fusion, we need to establish correspondence between keypoints detected by different methods. We employ a mutual nearest neighbor (MNN) algorithm with spatial tolerance.

\subsection{Algorithm Description}

Given two keypoint sets $K_A$ (e.g., SIFT-detected) and $K_B$ (e.g., KeyNet-detected), we compute the intersection as follows:

\begin{enumerate}
    \item Build KD-tree spatial indices for both sets: $T_A$, $T_B$

    \item For each keypoint $k_a \in K_A$:
    \begin{enumerate}
        \item Find nearest neighbor $k_b = \text{NN}(k_a, T_B)$
        \item Check forward tolerance: $\|k_a - k_b\|_2 \leq \tau$
        \item Find reverse nearest neighbor $k_a' = \text{NN}(k_b, T_A)$
        \item Check mutual agreement: $k_a' = k_a$
        \item Check reverse tolerance: $\|k_b - k_a'\|_2 \leq \tau$
        \item Check uniqueness: $k_b$ not already matched
        \item If all checks pass, add $(k_a, k_b)$ to intersection
    \end{enumerate}

    \item Output: Paired keypoint sets $(K_A^*, K_B^*)$ with 1-to-1 correspondence
\end{enumerate}

\subsection{Algorithm Properties}

The MNN algorithm guarantees several desirable properties:

\begin{itemize}
    \item \textbf{1-to-1 correspondence}: Each keypoint matched at most once
    \item \textbf{Symmetry}: Same result regardless of which set is processed first
    \item \textbf{Tolerance-based}: Configurable spatial acceptance threshold
    \item \textbf{Mutual agreement}: Both keypoints must be each other's nearest neighbor
\end{itemize}

\subsection{Tolerance Selection}

Following Mikolajczyk and Schmid's detector evaluation methodology, we use a default tolerance of $\tau = 3.0$ pixels. This value balances:

\begin{itemize}
    \item \textbf{Strict enough}: Prevents misaligned correspondences
    \item \textbf{Loose enough}: Accounts for typical detector variance (1-3 pixels)
    \item \textbf{Literature-supported}: Standard practice in feature matching research
\end{itemize}

\subsection{Implementation}

The intersection algorithm is implemented in the \texttt{keypoint\_manager} CLI tool:

\begin{verbatim}
./keypoint_manager build-intersection \
    --source-a sift_8000 \
    --source-b keynet_8000 \
    --out-a sift_intersection \
    --out-b keynet_intersection \
    --tolerance 3.0
\end{verbatim}

The resulting keypoint sets are stored in the database with full provenance tracking.

\section{Scale-Matching Strategy}
\label{sec:scale_matching}

We observed that SIFT and KeyNet detectors produce keypoints with dramatically different scale distributions. This motivated our scale-matching strategy.

\subsection{Scale Distribution Analysis}

Table~\ref{tab:scale_distribution} shows the scale characteristics of different keypoint sets:

\begin{table}[h]
\centering
\caption{Scale distribution of keypoint sets}
\label{tab:scale_distribution}
\begin{tabular}{lrrr}
\toprule
\textbf{Keypoint Set} & \textbf{Count} & \textbf{Mean Scale} & \textbf{Std Dev} \\
\midrule
sift\_8000 (full) & 2.5M & 4.45px & 3.2px \\
keynet\_8000 (full) & 2.8M & 49.83px & 28.1px \\
sift\_scale\_6px (filtered) & 645K & 10.03px & 4.1px \\
keynet\_scale\_6px (filtered) & 645K & 92.39px & 31.2px \\
\bottomrule
\end{tabular}
\end{table}

\subsection{Scale Filtering Methodology}

To create scale-controlled keypoint sets, we:

\begin{enumerate}
    \item Sort keypoints by scale (descending)
    \item Retain top $k$\% (typically 25\%)
    \item Apply minimum scale threshold (6 pixels) to exclude aliased features
\end{enumerate}

\subsection{Scale-Matched Intersection}

For cross-detector fusion, we combine scale filtering with spatial intersection:

\begin{enumerate}
    \item Generate scale-controlled SIFT keypoints: $K_A^{\text{scale}}$
    \item Generate scale-controlled KeyNet keypoints: $K_B^{\text{scale}}$
    \item Compute spatial intersection: $(K_A^*, K_B^*)$
\end{enumerate}

This produces aligned keypoint pairs where:
\begin{itemize}
    \item SIFT keypoints: Average 7.64 pixel scale
    \item KeyNet keypoints: Average 89.74 pixel scale
    \item Both sets: 111K paired keypoints with 1-to-1 correspondence
\end{itemize}

\section{Descriptor Fusion Methods}

With aligned keypoint sets, we implement two fusion strategies:

\subsection{Weighted Averaging}

\begin{equation}
    d_{\text{avg}} = \alpha \cdot d_A + (1 - \alpha) \cdot d_B
\end{equation}

where $\alpha \in [0, 1]$ controls the contribution of each descriptor. We use $\alpha = 0.5$ for equal weighting.

\textbf{Requirements}:
\begin{itemize}
    \item Same dimensionality: $\text{dim}(d_A) = \text{dim}(d_B)$
    \item Normalization: Both descriptors should be L2-normalized before averaging
\end{itemize}

\subsection{Concatenation}

\begin{equation}
    d_{\text{concat}} = [d_A, d_B]
\end{equation}

\textbf{Properties}:
\begin{itemize}
    \item Preserves both representations fully
    \item Doubles dimensionality: 128D + 128D = 256D
    \item No information loss from aggregation
\end{itemize}

\section{Experimental Pipeline}

Our experimental workflow integrates the components described above into a cohesive pipeline:

\begin{enumerate}
    \item \textbf{Detection}: Generate keypoints for all HPatches images using detectors (SIFT, KeyNet) via Python/Kornia scripts.
    \item \textbf{Intersection}: Compute spatially aligned keypoint subsets using the MNN algorithm ($\tau=3.0$px) and scale filtering.
    \item \textbf{Extraction}: Compute descriptors for these locked keypoints. SIFT/RootSIFT use OpenCV; HardNet/SOSNet use the LibTorch C++ wrapper.
    \item \textbf{Fusion}: (Optional) Combine descriptors via concatenation or averaging.
    \item \textbf{Matching \& Evaluation}: Perform retrieval and matching tasks, computing metrics against ground-truth homographies.
\end{enumerate}

\section{Evaluation Methodology}

We employ a multi-faceted evaluation strategy based on the protocols defined by Balntas et al. and expanded by Bojani\'{c} et al.

\subsection{Task 1: Image Matching (mAP)}

The primary metric for descriptor utility is Mean Average Precision (mAP) in an image matching context.

\textbf{Protocol}:
\begin{enumerate}
    \item For each image pair (reference $I_A$, target $I_B$):
    \item Match descriptors using the Second Nearest Neighbor (SNN) ratio test.
    \item A match $(d_A^i, d_B^j)$ is correct if the geometric projection error $||H \cdot k_A^i - k_B^j||_2 \leq \tau$.
    \item Compute Average Precision (AP) as the area under the Precision-Recall curve.
\end{enumerate}

We report \textbf{True Micro mAP}: An Information Retrieval (IR) style metric where we aggregate AP over all queries. Crucially, we enforce a \textbf{Single Ground Truth (R=1)} policy: for a given keypoint in $I_A$, there is at most one correct match in $I_B$ (the nearest neighbor in the intersection set).

\subsection{Task 2: Keypoint Verification}

Following Bojani\'{c} et al., verification tests the descriptor's ability to distinguish true correspondences from false ones.

\textbf{Binary Classification}:
\begin{itemize}
    \item \textbf{Positive pairs}: Spatially corresponding keypoints from sequence pairs.
    \item \textbf{Negative pairs}: Spatially non-corresponding keypoints (distractors).
\end{itemize}
We compute the Area Under the ROC Curve (AUC) to measure discriminative power independent of the nearest-neighbor search strategy.

\subsection{Task 3: Keypoint Retrieval}

This task evaluates the descriptor's ranking capability. For a query patch, the system must rank a database of target patches.

\textbf{Labels}:
\begin{itemize}
    \item $y=1$: The true geometric match.
    \item $y=0$: "Hard negatives" (patches from the same image but different locations).
    \item $y=-1$: "Easy negatives" (patches from different sequences/scenes).
\end{itemize}
This metric highlights whether a descriptor has learned semantically meaningful representations that separate true matches from visually similar but geometrically distinct structures.

\subsection{Aggregation and Breakdown}

To account for the diversity of the HPatches dataset, we report results in two aggregation modes:
\begin{itemize}
    \item \textbf{Micro Average}: Aggregates all queries globally. biases towards scenes with more keypoints.
    \item \textbf{Macro Average}: Computes metrics per scene, then averages across scenes. This ensures that texture-poor scenes (with fewer keypoints) contribute equally to the final score.
\end{itemize}

Results are further stratified by scene type:
\begin{itemize}
    \item \textbf{HP-V (Viewpoint)}: 59 sequences with significant geometric deformations.
    \item \textbf{HP-I (Illumination)}: 57 sequences with lighting changes (day/night, flash/no-flash).
\end{itemize}

\section{Experimental Design}

\subsection{Independent Variables}

Our experiments vary the following factors:

\begin{itemize}
    \item \textbf{Descriptor type}: SIFT, RootSIFT, DSP-SIFT, HardNet, SOSNet
    \item \textbf{Keypoint set}: Full, scale-controlled, intersection, scale-matched intersection
    \item \textbf{Fusion method}: None, averaging, concatenation
    \item \textbf{Fusion pairs}: SIFT+CNN, CNN+CNN
\end{itemize}

\subsection{Dependent Variables}

We measure:
\begin{itemize}
    \item Mean Average Precision (mAP) - primary metric
    \item HP-V and HP-I breakdown
    \item Keypoint verification AP (when enabled)
    \item Keypoint retrieval AP (when enabled)
\end{itemize}

\subsection{Experiment Execution}

Experiments are specified in YAML configuration files and executed via:

\begin{verbatim}
./experiment_runner config/experiments/experiment.yaml
\end{verbatim}

Results are automatically stored in the SQLite database with full parameter logging for reproducibility.


% ========== Chapter 4
% Chapter 4: Experiments and Results
\chapter{Experiments and Results}
\label{chap:results}

This chapter presents the experimental evaluation of descriptor fusion strategies on the HPatches benchmark. We organize the results into four main studies: baseline descriptor performance, scale control effects, spatial intersection analysis, and descriptor fusion efficacy.

% ===================================================================
\section{Experimental Setup}
\label{sec:experimental_setup}
% ===================================================================

\subsection{Dataset and Metrics}

We evaluate all experiments on the HPatches benchmark, which consists of 116 image sequences with ground-truth homographies. The dataset is divided into 59 viewpoint sequences (geometric transformations) and 57 illumination sequences (photometric changes).

As defined in Chapter~\ref{chap:methodology}, we primarily report \textbf{Mean Average Precision (mAP)} using the True Micro mAP definition (single ground truth per query). We further breakdown results by sequence type:
\begin{itemize}
    \item \textbf{HP-V}: Viewpoint sequences (measuring geometric invariance)
    \item \textbf{HP-I}: Illumination sequences (measuring photometric invariance)
\end{itemize}

% ===================================================================
\section{Baseline Descriptor Performance}
\label{sec:baseline_results}
% ===================================================================

We first establish baseline performance for traditional and learned descriptors using their native keypoint detectors without any scale filtering.

\begin{table}[h]
\centering
\caption{Baseline descriptor performance on full keypoint sets (SIFT ~2.5M, KeyNet ~2.8M)}
\label{tab:baselines}
\begin{tabular}{llrrr}
\toprule
\textbf{Descriptor} & \textbf{Keypoint Set} & \textbf{mAP} & \textbf{HP-V} & \textbf{HP-I} \\
\midrule
SIFT & sift\_8000 & 44.5\% & 45.9\% & 43.1\% \\
RootSIFT & sift\_8000 & 46.7\% & 46.2\% & 47.2\% \\
HardNet & keynet\_8000 & 64.5\% & 63.8\% & 65.3\% \\
SOSNet & keynet\_8000 & 64.3\% & 63.4\% & 65.2\% \\
\bottomrule
\end{tabular}
\end{table}

As expected, the learned descriptors (HardNet, SOSNet) significantly outperform SIFT, achieving approximately 20 percentage points higher mAP. SIFT shows a slight preference for viewpoint changes, while the CNN descriptors perform slightly better on illumination sequences.

% ===================================================================
\section{Impact of Scale Control}
\label{sec:scale_control}
% ===================================================================

One of the central hypotheses of this thesis is that keypoint scale is a dominant factor in matching performance. By filtering the keypoint sets to retain only the largest 25\% of features (Scale-Controlled sets), we observe dramatic performance improvements across all descriptor types.

\begin{table}[h]
\centering
\caption{Impact of Scale Control (filtering small keypoints)}
\label{tab:scale_control}
\begin{tabular}{llrrr}
\toprule
\textbf{Configuration} & \textbf{Metric} & \textbf{Full Set} & \textbf{Scale Filtered} & \textbf{Improvement} \\
\midrule
\textbf{SIFT} & mAP & 44.5\% & \textbf{62.8\%} & \textbf{+18.3\%} \\
& HP-V & 45.9\% & 65.7\% & +19.8\% \\
& HP-I & 43.1\% & 59.8\% & +16.7\% \\
\midrule
\textbf{HardNet} & mAP & 64.5\% & \textbf{78.1\%} & \textbf{+13.6\%} \\
& HP-V & 63.8\% & 76.9\% & +13.1\% \\
& HP-I & 65.3\% & 79.3\% & +14.0\% \\
\bottomrule
\end{tabular}
\end{table}

The results in Table~\ref{tab:scale_control} are striking. Simply removing small, unstable keypoints improves SIFT's performance by over 18 percentage points, bringing it close to the baseline performance of HardNet. For HardNet, scale control yields a 13.6\% gain, pushing it to 78.1\% mAP. This confirms that descriptor distinctiveness is strongly correlated with patch size.

% ===================================================================
\section{Impact of Spatial Intersection}
\label{sec:intersection_results}
% ===================================================================

Our fusion methodology relies on finding the spatial intersection of keypoints detected by SIFT and KeyNet. We analyzed whether this intersection step itself acts as a quality filter.

\begin{table}[h]
\centering
\caption{Performance of HardNet on different keypoint subsets}
\label{tab:intersection_analysis}
\begin{tabular}{lrrr}
\toprule
\textbf{Keypoint Set} & \textbf{mAP} & \textbf{HP-V} & \textbf{HP-I} \\
\midrule
Full KeyNet Set & 64.5\% & 63.8\% & 65.3\% \\
Scale-Controlled & 78.1\% & 76.9\% & 79.3\% \\
\textbf{Spatial Intersection} & \textbf{82.1\%} & \textbf{81.5\%} & \textbf{82.7\%} \\
\bottomrule
\end{tabular}
\end{table}

Table~\ref{tab:intersection_analysis} shows that features detected by \textit{both} detectors are of higher quality than those detected by KeyNet alone, even after scale filtering. The intersection set yields an additional 4.0\% improvement in mAP, suggesting that detector consensus is a powerful proxy for feature repeatability.

% ===================================================================
\section{Descriptor Fusion Results}
\label{sec:fusion_results}
% ===================================================================

We evaluated two fusion strategies: concatenation and weighted averaging. We tested these on two classes of pairings: Cross-Family (SIFT+CNN) and Intra-Family (CNN+CNN).

\subsection{Cross-Family Fusion (SIFT + HardNet)}

Contrary to our initial expectations, fusing SIFT with HardNet did not yield performance gains.

\begin{itemize}
    \item \textbf{HardNet Baseline (Intersection)}: 82.1\% mAP
    \item \textbf{SIFT + HardNet (Concatenation)}: 71.4\% mAP
\end{itemize}

This 10.7\% drop in performance indicates a fundamental incompatibility between the descriptor spaces. As discussed in Chapter 3, SIFT histograms are non-negative and sparse, while HardNet embeddings are dense and zero-centered. Concatenation creates a heterogeneous feature vector where the L2 distance is dominated by the SIFT component's scale, effectively corrupting the high-quality HardNet information.

\subsection{Intra-Family Fusion (HardNet + SOSNet)}

Fusing two learned descriptors proved highly effective. Table~\ref{tab:fusion_results} shows the results for fusing HardNet and SOSNet on the scale-matched intersection set.

\begin{table}[h]
\centering
\caption{CNN+CNN Fusion Results (Scale-Matched Intersection)}
\label{tab:fusion_results}
\begin{tabular}{llrrr}
\toprule
\textbf{Descriptor} & \textbf{Fusion} & \textbf{mAP} & \textbf{HP-V} & \textbf{HP-I} \\
\midrule
HardNet & None & 82.1\% & 81.5\% & 82.7\% \\
SOSNet & None & 81.9\% & 81.2\% & 82.5\% \\
HardNet + SOSNet & Weighted Avg & 92.3\% & 91.4\% & 93.2\% \\
\textbf{HardNet + SOSNet} & \textbf{Concat} & \textbf{93.4\%} & \textbf{92.6\%} & \textbf{94.2\%} \\
\bottomrule
\end{tabular}
\end{table}

\textbf{Key Finding:} The concatenation of HardNet and SOSNet achieves a state-of-the-art mAP of \textbf{93.4\%}. This represents an 11.3 percentage point improvement over the single best descriptor (HardNet).

The success of this fusion suggests that while both networks are trained on similar data, they learn complementary representations of the image patches. Concatenation preserves this distinct information, whereas averaging tends to dilute it slightly (92.3\% vs 93.4\%).

% ===================================================================
\section{Viewpoint vs. Illumination Analysis}
\label{sec:viewpoint_illumination}
% ===================================================================

Analyzing the breakdown of results provides insight into where these methods excel:

\begin{enumerate}
    \item \textbf{Traditional Methods}: SIFT benefits immensely from scale control on viewpoint sequences (+19.8\% HP-V vs +16.7\% HP-I), confirming that scale variance is a primary source of error for hand-crafted detectors in geometric tasks.
    \item \textbf{Learned Methods}: HardNet and SOSNet are naturally robust to illumination changes (HP-I $>$ HP-V).
    \item \textbf{Fusion Synergy}: The fused CNN descriptor achieves excellent performance on both tasks (92.6\% HP-V, 94.2\% HP-I), effectively closing the gap between geometric and photometric invariance.
\end{enumerate}

\section{Summary}

Our experiments demonstrate three key conclusions:
\begin{enumerate}
    \item \textbf{Scale Matters}: Filtering for larger scales improves performance by 13-18\% across all descriptor types.
    \item \textbf{Consensus Matters}: Spatial intersection of detectors acts as a high-quality filter, boosting mAP by a further 4\%.
    \item \textbf{Compatible Fusion Works}: While cross-family fusion fails due to distribution mismatch, fusing compatible learned descriptors (HardNet+SOSNet) yields significant gains, achieving a peak mAP of 93.4\%.
\end{enumerate}

% ========== Chapter 5
\chapter{Discussion}
\label{chap:discussion}
% Chapter 5: Discussion
% CLAUDE EDIT: New discussion chapter 2025-12-07

This chapter interprets our experimental findings, explaining the underlying causes of observed performance patterns and providing practical recommendations.

\section{Why Scale Control Matters}

Our experiments demonstrate that scale control yields dramatic improvements: +39\% relative for SIFT-family descriptors and +21\% relative for CNN descriptors. We attribute this to several factors:

\subsection{Information Content}

Larger-scale keypoints capture patches containing more pixels:
\begin{itemize}
    \item A 4-pixel scale keypoint samples approximately a 16$\times$16 pixel region
    \item A 10-pixel scale keypoint samples approximately a 40$\times$40 pixel region
    \item Larger regions contain more distinctive texture and edge information
\end{itemize}

\subsection{Aliasing and Noise}

Small-scale keypoints are more susceptible to:
\begin{itemize}
    \item \textbf{Aliasing}: High-frequency content that violates Nyquist sampling
    \item \textbf{Noise sensitivity}: Small patches have lower signal-to-noise ratio
    \item \textbf{Localization error}: Sub-pixel errors have greater relative impact
\end{itemize}

\subsection{Practical Recommendation}

For production systems requiring high matching accuracy, we recommend filtering to the top 25\% of keypoints by scale. The trade-off is reduced keypoint count (645K vs 2.5M in our experiments), but the quality improvement substantially outweighs the coverage reduction.

\section{Understanding Fusion Failures}

\subsection{Distribution Incompatibility}

Our analysis reveals that SIFT+CNN fusion fails due to fundamental distribution incompatibility:

\begin{table}[h]
\centering
\caption{Descriptor distribution comparison}
\begin{tabular}{lcc}
\toprule
\textbf{Descriptor} & \textbf{Range} & \textbf{Mean} \\
\midrule
SIFT & [0, 0.3] & 0.05 \\
HardNet & [-0.3, +0.3] & 0.0 \\
\bottomrule
\end{tabular}
\end{table}

When averaging a SIFT value $s \approx 0.05$ with a HardNet value $h \approx 0$:
\begin{equation}
    \text{avg} = \frac{s + h}{2} \approx 0.025
\end{equation}

The result is shifted from the zero-centered HardNet distribution, effectively corrupting the learned representation.

\subsection{Fusion Contribution Analysis}

% Reference to fusion_contribution_analysis.png
Our correlation analysis shows that in SIFT+HardNet fusion:
\begin{itemize}
    \item HardNet correlates 0.9 with the fused descriptor
    \item SIFT correlates only 0.4 with the fused descriptor
\end{itemize}

This asymmetry indicates that averaging does not produce a balanced combination---SIFT's non-negative values shift the mean, corrupting HardNet's zero-centered representation.

\subsection{Why Concatenation Also Fails for SIFT+CNN}

Even concatenation produces degraded results (71.4\% vs 78.1\% HardNet alone). This occurs because:
\begin{enumerate}
    \item L2 distance weights all dimensions equally
    \item SIFT dimensions (always positive) contribute large positive values
    \item HardNet dimensions (centered at zero) contribute smaller absolute values
    \item SIFT effectively dominates the distance calculation
\end{enumerate}

\section{Why CNN+CNN Fusion Succeeds}

HardNet+SOSNet concatenation achieves 93.4\% mAP, improving upon either descriptor alone. We attribute this success to:

\subsection{Distribution Compatibility}

Both descriptors are:
\begin{itemize}
    \item Zero-centered (mean $\approx$ 0)
    \item Similar value ranges ([-0.3, +0.3])
    \item L2-normalized
\end{itemize}

This compatibility ensures that averaging or concatenation produces meaningful combinations.

\subsection{Complementary Features}

Despite similar training methodologies, HardNet and SOSNet learn slightly different representations:
\begin{itemize}
    \item \textbf{HardNet}: Trained with hard negative mining, focuses on discriminative features
    \item \textbf{SOSNet}: Incorporates second-order similarity, captures different geometric relationships
\end{itemize}

Concatenation preserves both representations, allowing the matcher to utilize all available information.

\subsection{Why Concatenation Outperforms Averaging}

Concatenation (+0.5\%) outperforms averaging (-0.6\%) because:
\begin{enumerate}
    \item Averaging collapses 256 dimensions of information into 128
    \item Complementary features may be lost in averaging
    \item Concatenation preserves full information from both descriptors
\end{enumerate}

\section{Detector Agreement as Quality Signal}

Keypoints detected by both SIFT and KeyNet (spatial intersection) achieve higher performance than either detector alone:
\begin{itemize}
    \item HardNet on full KeyNet: 64.5\% mAP
    \item HardNet on intersection: 82.1\% mAP (+17.6\%)
\end{itemize}

This improvement suggests that detector agreement provides a quality signal: keypoints found by both methods likely correspond to genuinely salient image features, while keypoints found by only one detector may represent detector-specific artifacts.

\section{HP-V vs HP-I Patterns}

\subsection{Traditional vs Learned Preferences}

Our results show opposite patterns:
\begin{itemize}
    \item \textbf{Traditional descriptors}: HP-V $>$ HP-I (better on viewpoint changes)
    \item \textbf{CNN descriptors}: HP-I $>$ HP-V (better on illumination changes)
\end{itemize}

\subsection{Explanation}

CNN descriptors are trained on patches with photometric augmentations (brightness, contrast, gamma), leading to learned illumination invariance. Traditional descriptors rely on gradient directions, which are naturally invariant to multiplicative illumination changes but not to more complex photometric transformations.

\section{Limitations}

\subsection{Dataset Scope}

Our experiments use only the HPatches benchmark. While HPatches is a standard evaluation dataset, results may not generalize to:
\begin{itemize}
    \item Extreme viewpoint changes ($>$ 60 degrees)
    \item Significant scale differences between images
    \item Different image domains (medical, satellite, etc.)
\end{itemize}

\subsection{Computational Considerations}

Concatenation doubles descriptor dimensionality (128D to 256D), increasing:
\begin{itemize}
    \item Memory requirements
    \item Matching time (linear in dimensionality for brute-force)
    \item Index size for approximate nearest neighbor methods
\end{itemize}

For real-time applications, this overhead may be prohibitive.

\subsection{Detector Dependency}

Our best results use KeyNet for CNN descriptors. The findings may not transfer to other detector choices (e.g., SuperPoint's built-in detector).

\section{Summary of Insights}

\begin{enumerate}
    \item \textbf{Scale matters more than descriptor choice}: A 39\% improvement from scale control exceeds most algorithmic improvements
    \item \textbf{Distribution compatibility is essential}: SIFT+CNN fusion fails due to incompatible value distributions
    \item \textbf{Concatenation $>$ averaging}: Preserving full information outperforms lossy aggregation
    \item \textbf{Detector agreement filters quality}: Intersection keypoints are more reliable than single-detector keypoints
    \item \textbf{CNN+CNN fusion works}: When distributions are compatible, complementary features improve performance
\end{enumerate}


% ========== Chapter 6
\chapter{Conclusion and Future Work}
\label{chap:conclusion}
% Chapter 6: Conclusion and Future Work
% TODO: Remove DRAFTING comments before final submission

\section{Summary of Contributions}

This thesis investigated detector consensus and descriptor fusion for local feature matching, with the following contributions:

\begin{enumerate}
    \item \textbf{Detector Intersection as Quality Filter}: Keypoints detected by multiple distinct detectors are more distinctive than those found by any single detector. This effect holds for both SIFT-SURF and SIFT-KeyNet intersections, with HardNet achieving 82.4\% mAP on intersection keypoints---a 25\% relative improvement.

    \item \textbf{Color HPatches Benchmark}: We created a color version of the HPatches patch benchmark by re-extracting 65$\times$65 color patches from original images. This enables evaluation of color descriptors like HoNC.

    \item \textbf{Magnitude Matching for Cross-Family Fusion}: Cross-family fusion (SIFT+CNN) requires pre-fusion L2 normalization to ensure equal contribution from each descriptor. With proper normalization, SIFT+HardNet achieves 46.0\% mAP on patches.

    \item \textbf{Complementary Descriptor Fusion}: Pairing a ``discriminator'' (HoNC, high verification-to-matching ratio) with a ``matcher'' (CNN) yields substantial improvements. HoNC+SOSNet concatenation achieves 50.6\% mAP on the color patch benchmark, outperforming all individual descriptors.

    \item \textbf{Scale Control Methodology}: Filtering keypoints by scale yields large improvements: +39\% relative for SIFT-family descriptors and +21\% relative for CNN descriptors.

    \item \textbf{DescriptorWorkbench Framework}: We developed an open-source evaluation framework implementing image matching, keypoint verification, and keypoint retrieval metrics from Bojanic et al.~\cite{bojanic2020comparison}, supporting both full-image and patch-based evaluation.
\end{enumerate}

\section{Key Findings}

Our experiments reveal several insights for local feature matching:

\begin{itemize}
    \item \textbf{Quality over quantity}: Fewer, better keypoints (scale-filtered or intersection-filtered) outperform larger sets of lower-quality keypoints.

    \item \textbf{Detector agreement provides quality signal}: Keypoints detected by multiple distinct methods are more reliable, with 17--25\% improvements from intersection filtering.

    \item \textbf{Magnitude matching enables cross-family fusion}: Pre-fusion L2 normalization allows successful SIFT+CNN combination by ensuring equal contribution from each descriptor.

    \item \textbf{Complementary descriptors outperform similar ones}: HoNC+CNN fusion outperforms CNN+CNN or SIFT+SIFT because the descriptors capture different information (color vs. learned features).

    \item \textbf{Concatenation preserves information}: Concatenation consistently outperforms averaging by preserving complementary features.
\end{itemize}

\section{Practical Recommendations}

Based on our findings, we offer the following recommendations:

\begin{enumerate}
    \item \textbf{For best patch matching}: Use HoNC+SOSNet concatenation (50.6\% mAP), leveraging color discrimination with learned matching.

    \item \textbf{For best full-image matching}: Use HardNet on detector intersection keypoints (82.4\% mAP), or HardNet+SOSNet concatenation (93.4\% mAP) for highest accuracy.

    \item \textbf{For cross-family fusion}: Always L2 normalize each descriptor component before fusion to ensure equal contribution.

    \item \textbf{For traditional descriptors}: Apply scale filtering (top 25\%) and use intersection keypoints (64--75\% mAP vs 42\% baseline).
\end{enumerate}

\section{Future Work}

Several directions remain for future investigation:

\subsection{Learned Fusion Weights}

Rather than fixed $\alpha = 0.5$ weighting, learn optimal weights:
\begin{equation}
    d_{\text{fused}} = \sum_i w_i \cdot d_i, \quad \text{s.t.} \sum_i w_i = 1
\end{equation}
Weights could be learned per-dimension or globally, potentially improving cross-family fusion further.

\subsection{Additional Datasets}

Validate findings on:
\begin{itemize}
    \item Oxford5k and Paris6k (image retrieval)
    \item MegaDepth (wide-baseline matching)
    \item ETH3D (multi-view stereo)
\end{itemize}

\subsection{End-to-End Learning}

Train a joint detector-descriptor-fusion network that:
\begin{itemize}
    \item Detects keypoints with scale optimization
    \item Extracts multiple descriptor types
    \item Learns optimal fusion strategy
\end{itemize}

\subsection{Tolerance Sensitivity}

% Note: Results from overnight experiments will inform this
Systematically study the relationship between intersection tolerance and matching performance. Our hypothesis is an inverted U-curve: too strict yields few keypoints, too loose yields misaligned pairs.

\section{Closing Remarks}
% DRAFTING: Updated to reflect detector consensus and color fusion findings

This thesis demonstrates two complementary approaches to improving local feature matching. First, using multiple detectors as a consensus filter identifies high-quality keypoints that are more distinctive and repeatable than those found by any single detector. Second, fusing complementary descriptors---particularly color descriptors like HoNC with learned descriptors like SOSNet---captures information that neither descriptor provides alone.

A key technical finding is that cross-family fusion (SIFT+CNN) requires proper magnitude matching through pre-fusion L2 normalization. The initial fusion failures were not due to fundamental incompatibility between descriptor families, but rather a fixable magnitude mismatch problem.

More broadly, our findings suggest that keypoint quality deserves more attention. The 39\% improvement from scale control and 25\% improvement from detector consensus exceed many algorithmic advances, yet these filtering strategies are rarely discussed in the literature. We hope this work encourages more research into keypoint selection strategies alongside descriptor algorithm development.

The DescriptorWorkbench framework, color HPatches benchmark, and all experimental configurations are available as open-source software, enabling reproduction and extension of these results.


%
% ==========   Bibliography
%
\nocite{*}   % include everything in the uwthesis.bib file
\bibliographystyle{plain}
\bibliography{uwthesis}

\end{document}
