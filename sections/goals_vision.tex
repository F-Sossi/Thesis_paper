\section{Goals/Vision}

The overall goal of this project is to bridge the existing knowledge gap in the realm of computer vision, particularly focusing on the pooling of keypoint descriptors to enhance the performance of descriptor representations in image recognition tasks. The following are the primary objectives and vision of this research:

\begin{itemize}
    \item \textbf{Understanding Descriptor Pooling:} Conduct research into the implementation of descriptor pooling as opposed to or in conjunction with descriptor stacking, to generate a robust and space efficient descriptor representation. This method aims to reduce dimensionality and achieve invariance to transformations by aggregating features or descriptors over regions of an image\cite{dong2015domain}.

    \item \textbf{Implementation and Optimization:} Implement, optimize, and benchmark a variety of established and experimental descriptors under different pooling configurations. This includes exploring innovative optimization processes to augment the efficacy of pooled descriptors, with a potential stretch goal of leveraging advancements in GPU and tensor cores for enhanced computational efficiency.

    \item \textbf{Performance Evaluation:} Conduct an extensive bench-marking exercise to evaluate the performance of pooled descriptors, particularly focusing on precision and recall metrics to assess the accuracy and completeness of the descriptor matching process.

    \item \textbf{Contribution to Computer Vision:} By investigating the hierarchical performance outcomes of descriptor pooling, this research aims to enhance practical methodologies and applications in the field of computer vision.

    \item \textbf{Practical Implications:} The anticipated findings from this research could serve as a foundation for developing more advanced descriptor pooling strategies, thereby contributing to the broader field of computer vision, and potentially leading to enhanced accuracy and reliability in computer vision tasks.
\end{itemize}

The problem at hand is the lack of a thorough understanding concerning the performance and efficiency outcomes when various descriptors are pooled. The beneficiaries of this research span academia, where the findings could inform future research in computer vision, and the industry, particularly sectors reliant on image recognition technologies for various applications such as autonomous vehicles, robotics, and security systems.

\subsection{Stakeholders and Beneficiaries}
The primary stakeholders of this research include the academic community focusing on computer vision and related fields, and industries that heavily rely on image recognition technologies. The beneficiaries extend to sectors like automotive for autonomous driving, robotics for enhanced object recognition, and security systems for better surveillance and monitoring.
